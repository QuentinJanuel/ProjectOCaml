\documentclass[a4paper, 12pt]{article}
\usepackage[latin1]{inputenc}
\usepackage[T1]{fontenc}
\usepackage[french]{babel}
\usepackage{graphicx}
\usepackage{amsmath}
\usepackage{amssymb}

\usepackage{hyperref}

\pagestyle{headings}

\title{Titre}
\author{Auteurs}
\date{\today}

\begin{document}

\maketitle

\newpage

\section{Une premi�re partie}

\subsection{Une sous-partie}

Une liste :
\begin{itemize}
\item une chose ;
\item deux.
\end{itemize}

Une liste �num�r�e :
\begin{enumerate}
\item une chose ;
\item deux.
\end{enumerate}

\begin{verbatim}
du code source
\end{verbatim}

\section{Une deuxi�me partie}

Des maths : $a_0, \ldots, a_n$

$$
(a+b)^2 = a^3+3a^2b+3ab^2+b^3  
$$

Un tableau :
\begin{table}[htbp]
  \centering
  \begin{tabular}{||l|c|r|}\hline
    \textbf{Produit} & \textbf{Prix par} & \textbf{Prix en euro}\\\hline\hline
    Diamant          & carat             & 5000                 \\\hline
    Topaze           & gramme            & 2500                 \\\hline
    Banane           & kilo              & 2                    \\\hline
    Ferrari          & unit�             & 90000                \\\hline
  \end{tabular}
  \caption{Une l�gende}
\end{table}


\section{Insertion d'images}

L'insertion d'image se fait au moyen de la commande \texttt{includegraphics} du package \texttt{graphicx}. Si vous compilez avec \texttt{pdflatex}, vous ne pouvez inclure que des images au format png, jpg et pdf.
Voir le code source de ce fichier pour un exemple d'insertion d'image.
%\begin{center}
%  \includegraphics[scale=0.5]{image.png}
%\end{center}


\end{document}

